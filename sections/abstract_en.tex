\chapter{Abstract}
\label{ch:abstract_en}

Magnetic separation is a method for the selective separation and concentration of magnetic materials based on the differences in magnetic properties between particles. Despite continued progress a multidimensional separation of magnetic nanoparticles still constitutes a major challenge. In this work a new approach for the separation of magnetic nanoparticles with a multicolumn chromatography system is investigated. In order to identify the influence of different parameters on the separation process an in silico model simulating the interactions between a simplified packed bed and a suspension of magnetic nanoparticles was developed and validated. With this model parameter studies were conducted for different conditions. It is shown that higher magnetic field strenghts lead to a size dependent retention of the nanoparticles in the vicinity of the matrix material. It can also be demonstrated, that a decrease in the size of the matrix material elements results in an increase of the retention and size separation effect. The flow rate on the contrary does not have a significant impact on the residence time in the tested regime.  

In addition, the different retention of nanoparticles depending on the their size and magnetic properties within a magnetized packed bed is investigated experimentally in a one column setup. It is demonstrated that the observed retention of the nanoparticles is solely caused by the magnetized packed bed and is not produced by the column packing or setup. The performance of the system was evaluated for several different sized nanoparticle suspensions and different concentrations. Furthermore a limited ability to separate two nanoparticle suspensions of different sizes with the separation column is demonstrated. This indicates that a complete size separation of the nanopaticle suspensions should be possible with the multicolumn chromatography system. 