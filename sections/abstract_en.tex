\chapter{Acknowledgements}
\label{ch:Acknowledgements}


\newpage

\chapter{Abstract}
\label{ch:abstract_en}

Despite continued progress, the separation of magnetic nanoparticles still constitutes a major challenge. Especially for the continuous size fractionation of particles smaller than 1\,\textmu m no method exits, which would allow a multidimensional separation with narrow separation limits. In this work, first steps of a novel approach for the separation of magnetic nanoparticles with a multicolumn chromatography system are investigated. In order to identify the influence of different parameters on the separation process, an in silico model simulating the interactions between a simplified packed bed and a suspension of magnetic nanoparticles was developed and validated. With this model parameter studies were conducted for different conditions. It is shown that higher magnetic field strenghts lead to a size dependent retention of the nanoparticles in the vicinity of the matrix material. It is also demonstrated, that a decrease in the size of the matrix material elements results in an increase of the retention and size separation effect. The flow rate on contrary does not have a significant impact on the residence time in the tested regime.  

In addition, the different retention of nanoparticles depending on their size and magnetic properties within a magnetized packed bed is investigated experimentally in a single column setup. The retention and size separation effects observed in the simulations are confirmed by the experimental results. It is demonstrated that the observed retention of the nanoparticles is solely caused by the magnetized packed bed and is not due to the column packing or setup. The performance of the system was evaluated for several differently sized nanoparticle suspensions and different concentrations. Furthermore, a limited ability to separate two nanoparticle suspensions of different sizes with the separation column is demonstrated. Therefore a multicolumn system with the examined separation columns represents a promising approach for the selective size separation of magnetic nanoparticles.

%This indicates that a complete size separation of the nanoparticle suspensions should be possible with the multicolumn chromatography system. 

%Magnetic separation is a method for the selective separation and concentration of magnetic materials based on the differences in magnetic properties between particles.