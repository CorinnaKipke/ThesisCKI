%% LaTeX2e class for student theses
%% sections/content.tex
%% 
%% Karlsruhe Institute of Technology
%% Institute for Program Structures and Data Organization
%% Chair for Software Design and Quality (SDQ)
%%
%% Dr.-Ing. Erik Burger
%% burger@kit.edu
%%
%% Version 1.1, 2014-11-21

\chapter{Introduction and Motivation}
\label{ch:Introduction}
%experimental cancer treatment called magnetic hyperthermia
%Magnetic nanotechnologies and especially magnetic nanoparticles and their applications in both reasearch and industry
The selective fractionation of ultrafine particles on an industrial scale still constitutes a major challenge \cite{mandel2012magnetic}. Especially for the continuous size fractionation of particles smaller than 1\,\textmu m no method exits which would allow a multidimensional separation with narrow separation limits. Magnetic nanoparticles with a narrow size distribution are widely used in a broad range of applications e.g. as part of high frequency antennas and forgery-proof ink. Also biomedical applications using magnetic nanoparticles like an experimental cancer treatment called magnetic hyperthermia, drug therapies for tissue specific targeting or contrast agents in radiological imaging have been developed recently \cite{mohammed2017magnetic}. Those processes are heavily dependant on the availability of nanoparticles with a narrow particle size range for a reasonable price.   
%Magnetic separation is a technique to selectively separate and concentrate magnetic materials based on the differences in magnetic properties between particles \cite{ge2017magnetic}. 
%Since the 19\textsuperscript{th} century \textbf{Markus:, (glaub ich)} magnetic separation has been used in many industrial processes \textbf{Markus:das bedeutet 1800-1899 (nur zur Kontrolle)}. Already in the sixth century BC magnetic behaviour of materials was described, \textbf{Markus:semicolon oder Punkt hier} the cause of the magnetic force, however, was not discovered until the late 18\textsuperscript{th} century \cite{livingston1996driving}\textbf{Markus:für meinen Geschmack ist das ein bisschen viel Geschichtslehrstunde}. The mathematical framework for the description of electricity and magnetism developed by among others Helmholtz, Gauss and Faraday enabled a better understanding of the forces acting on magnetic materials such as lodestones \cite{yavuz2009magnetic}. 
%The first application of magnetic separation can be dated back to 1792, when William Fullarton filed a patent describing the use of magnets for the separation of iron minerals \cite{1794repertory}. From then on magnetic separation methods were regularly applied in the chemical and mining industries. More recently, applications in waste water treatment, food industry and biotechnology have become available \cite{yavuz2009magnetic}. Magnetic separation is based on the simple principle that materials with different magnetic properties experience different magnetic forces in the presence of a magnetic field gradient generated by an externally applied field \cite{svoboda2004magnetic}. With conventional magnetic separators only ferromagnetic materials with a particle size larger than approximately 50\,\textmu m can be separated. High gradient magnetic separation (HGMS), however, enables the magnetic separation of smaller and only weakly magnetic particles \cite{frantz1937patent,ge2017magnetic}. This is achieved through the combination of strong magnetic fields and high magnetic field gradients within a separation matrix. 

Although there have been a few studies concerning the chromatographic separation of magnetic nanoparticles by field-flow fractionation with a high gradient static magnetic field, a multidimensional separation method for mixtures of magnetic particles smaller than 200\,nm with moderate to high throughput has not yet been established \cite{kim2007development,williams2009magnetic,williams2010characterization}. A possible method for the size separation of magnetic nanoparticles was proposed by \cite{nomizu1996magnetic} using a magnetized packed bed to generate high magnetic-field gradients in addition to short distances between the matrix material and the nanoparticles. This method, however, was not further investigated for its size separation efficiency on nanoparticles. A new approach for the separation of magnetic nanoparticles is a multicolumn chromatography system, which takes advantage of the different magnetic properties of particles with different sizes. For this approach an in-depth knowledge of the participating forces is required.    
In the 1980s, several different models for the calculation of the concentration distribution of magnetic nanoparticles around magnetized wires \cite{gerber1984magnetic,davies19902,fletcher1991fine} and spheres \cite{moyer1986filtration} due to a magnetic field gradient were developed. The goal of those models, however, was to predict the separation efficiency and application limits of the classical high gradient magnetic separators and not to investigate the dynamic processes associated with a particle size dependent dynamic retention in a magnetized packed bed.

This thesis describes the design and implementation of an in silico model simulating the interactions between a simplified packed bed and a suspension of magnetic nanoparticles. The main goal of the simulation is to investigate the influences of different parameters like the particle size, flow velocity or magnetic field strength on the retention behaviour of the magnetic nanoparticles. For this, parameter studies for different conditions were conducted. In addition, the different retention of nanoparticles depending on the their size and magnetic properties within a magnetized packed bed is investigated experimentally. As proof of principle a buffer-exchange of the nanoparticle suspension within a separation column is demonstrated. Furthermore a separation of two different sized nanoparticle suspensions is attempted with the single column setup.
