%% LaTeX2e class for student theses
%% sections/content.tex
%% 
%% Karlsruhe Institute of Technology
%% Institute for Program Structures and Data Organization
%% Chair for Software Design and Quality (SDQ)
%%
%% Dr.-Ing. Erik Burger
%% burger@kit.edu
%%
%% Version 1.1, 2014-11-21

\chapter{Introduction}
\label{ch:Introduction}

The properties of magnetic materials were identified as early
as the sixth century BC, but the means by which magnets could
move material remained only a curious phenomenon until the late
18th century (Livingston, 1997). As Gauss, Helmholtz and others
developed a framework for electricity and magnetism, the reasons
that magnets could move materials such as lodestone became ap-
parent. This once mysterious force was quickly put to use in the
nascent chemical and mining industries. \cite{yavuz2009magnetic}

The first application of magnetic separation can be dated back to 1792, when William Fullarton filed a patent describing the use of magnets for the separation of iron minerals \cite{1794repertory}. 

\section{State of the art}

\section{Objectives}