%% LaTeX2e class for student theses
%% sections/content.tex
%% 
%% Karlsruhe Institute of Technology
%% Institute for Program Structures and Data Organization
%% Chair for Software Design and Quality (SDQ)
%%
%% Dr.-Ing. Erik Burger
%% burger@kit.edu
%%
%% Version 1.1, 2014-11-21

\chapter{Introduction and motivation}
\label{ch:Introduction}
Magnetic separation is a technique to selectively separate and concentrate magnetic materials based on the differences in magnetic properties between particles \cite{ge2017magnetic}. Since the 19\textsuperscript{th} century magnetic separation has been used in many industrial processes. Already in the sixth century BC magnetic behaviour of materials was describe, the cause of the magnetic force, however, was not discovered until the late 18\textsuperscript{th} century \cite{livingston1996driving}. The mathematical framework for the description of electricity and magnetism developed by among others Helmholtz, Gauss and Faraday enabled a better understanding of the forces acting on magnetic materials such as lodestones \cite{yavuz2009magnetic}. 

The first application of magnetic separation can be dated back to 1792, when William Fullarton filed a patent describing the use of magnets for the separation of iron minerals \cite{1794repertory}. From then on magnetic separation methods were regularly applied in the chemical and mining industries. More recently, applications in waste water treatment, food industry and biotechnology have become available \cite{yavuz2009magnetic}. Magnetic separation is based on the simple principle that materials with different magnetic properties experience different magnetic forces in the presence of a magnetic field gradient generated by an externally applied field \cite{svoboda2004magnetic}. With conventional magnetic separators only ferromagnetic materials with a particle size larger than approximately 50\,\textmu m can be separated. High gradient magnetic separation (HGMS), however, enables the magnetic separation of smaller and only weakly magnetic particles \cite{frantz1937patent,ge2017magnetic}. This is achieved through the combination of strong magnetic fields and high magnetic field gradients within a separation matrix. The separation of magnetic particles in the nanometer range, however, still constitutes a real challenge \cite{mandel2012magnetic}. Magnetic nanoparticles with a narrow size distribution are widely used in a broad range of applications for example as contrast agent or carrier particles in medical applications or as part of high frequency antennas and forgery-proof ink. Although there have been a few studies concerning the chromatographic separation of magnetic nanoparticles by field-flow fractionation with a high gradient static magnetic field, a multidimensional separation method for mixtures of magnetic particles smaller than 200\,nm and a high throughput has not yet been established \cite{kim2007development,williams2009magnetic,williams2010characterization}. A possible method for the size separation of magnetic nanoparticles was proposed by \cite{nomizu1996magnetic} using a magnetized packed be to generate high magnetic-field gradients in addition to short distances between the matrix material and the nanoparticles. This method, however, was not further investigated for its size separation efficiency on nanoparticles. A new approach for the separation of magnetic nanoparticles is a multicolumn chromatography system, which takes advantage of the different magnetic properties of particles with different sizes. For this approach a in-depth knowledge of the participating forces is required.    

In the 1980s several different models for the calculation of the concentration distribution of magnetic nanoparticles around magnetized wires \cite{gerber1984magnetic,davies19902,fletcher1991fine} and spheres \cite{moyer1986filtration} due to a magnetic field gradient were developed. The goal of those models, however, was to predict the separation efficiency and boundary of the classical HGMS separators and not to investigate the dynamic processes associated with a particle size separation in a magnetized packed.

\section{Objectives}
\label{sec:Objectives}

This thesis describes the design and implementation of an in silico model simulating the interactions between a simplified packed bed and a suspension of magnetic nanoparticles. The main goal of the simulation was to investigate the influences of different parameters like the particle size, flow velocity or magnetic field strength on the retention behaviour of the magnetic nanoparticles. For this parameter studies for different conditions were conducted. In addition the different retention of nanoparticles depending on the their size and magnetic properties within a magnetized packed bed should be investigated experimentally. As proof of principle a buffer-exchange of the nanoparticle suspension with a separation column should be demonstrated. 

