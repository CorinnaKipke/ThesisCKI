\chapter{Zusammenfassung}
\label{ch:abstract_de}
Die Magnetseparation ist eine Methode, die eine selektiven Trennung und Aufkonzentrierung von magnetischen Materialien anhand ihrer unterschiedlichen magnetischen Eigenschaften ermöglicht. Trotz kontiuerlichem Fortschritt konnte noch keine Methode für die multidimensionale Trennung von magnetischen Nanopartikeln realisiert werden. In dieser Arbeit wird ein neuer Ansatz für die Trennung von magnetischen Feinstpartikeln mittels einer magnetisch induzierten Mehrsäulenchromatografie untersucht. Um den Einfluss verschiedener Parameter auf den Trennprozess zu analysieren wurde ein in silico Modell erstellt und validiert, welches die Wechselwirkungen zwischen einem vereinfachten gepackten Bett und einer Nanopartikelsuspension beschreibt. Mit diesem Modell wurden für unterschiedliche Bedingungen Parameterstudien durchgeführt. Höhere Magnetfeldstärken führen zu einer größenabhängigen Retention der Nanopartikel in der Nähe des Matrixmaterials. Die Ergebnisse zeigten weiterhin, dass ein Verkleinerung des Matrixmaterials zu einer höheren Retention und einer verbesserten Größentrennung führt. Die FLussrate hatte dagegen im getesteten Berich keinen wesentlichen Einfluss auf das Retentionsverhalten der Nanopartikel.

Des Weiteren wurde die Retention magneticher Feinstpartikeln
in einem magnetisierten, gepackten Bett in Abhängigkeit von deren Größe und magnetischer Eigenschaften experimentell untersucht. Die Versuche wurde mit einer Trennsäule in einem chromatographischen System durchgeführt. Es konnte gezeigt werden, dass die beobachtete Retention der Nanopartikel ausschließlich durch die magnetisierung des Matrixmaterials erzeugt wird. Es wurden keine Größentrennung aufgrund der Bettpackung oder des Systems festgestellt. Der Aufbau wurde mit unterschiedlich großen Nanopartikeln und Konzentrationen getestet und evaluiert. Zusätzlich wird eine Trennung von zwei unterschiedlich großen Magnetfluid-Lösungen untersucht. Eine vollständige Trennung der zwei Partikelgrößen konnte in dem Ein-Säulen-System nicht erreicht werden. Die Versuche legen jedoch nahe, dass eine Größentrennung mit einer Mehrsäulenchromatografie möglich sein sollte.  