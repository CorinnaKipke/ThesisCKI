\chapter{Zusammenfassung}
\label{ch:abstract_de}
Trotz anhaltendem Fortschritt stellt die selektive Trennung von magnetischen Nanopartikeln noch immer eine große Herausforderung dar. Insbesondere für die kontinuierliche Größenfraktionierung von Partikeln kleiner als 1\,\textmu m  konnte noch keine Methode für die multidimensionale Trennung von magnetischen Nanopartikeln mit engen Trenngrenzen realisiert werden. In dieser Arbeit werden erste Schritte eines neuen Ansatzes für die Trennung von magnetischen Feinstpartikeln mittels einer magnetisch induzierten Mehrsäulenchromatographie untersucht. Um den Einfluss verschiedener Parameter auf den Trennprozess zu analysieren wurde ein in silico Modell erstellt und validiert, welches die Wechselwirkungen zwischen einem vereinfachten gepackten Bett und einer Nanopartikelsuspension beschreibt. Mit diesem Modell wurden für unterschiedliche Bedingungen Parameterstudien durchgeführt. Höhere Magnetfeldstärken führen zu einer größenabhängigen Retention der Nanopartikel in der Nähe des Matrixmaterials. Die Ergebnisse zeigen weiterhin, dass ein abnehmender Radius der Matrixelemente zu einer höheren Retention und einer verbesserten Größentrennung führt. Die Flussrate hat dagegen im getesteten Bereich keinen wesentlichen Einfluss auf das Retentionsverhalten der Nanopartikel.

Des Weiteren wurde die Retention magnetischer Feinstpartikel
in einem magnetisierten, gepackten Bett in Abhängigkeit von deren Größe und magnetischen Eigenschaften experimentell untersucht. Die Versuche wurden mit einer Trennsäule in einem chromatographischen System durchgeführt. Die in den Simulationen beobachteten Retentions- und Größentrennungseffekte konnten durch die experimentellen Ergebnisse bestätigt werden. Es konnte gezeigt werden, dass die beobachtete Retention der Nanopartikel ausschließlich durch die Magnetisierung des Matrixmaterials erzeugt wird, da bei einer unmagnetischen Säulenpackung keine Retention auftritt. Der Aufbau wurde mit unterschiedlich großen Nanopartikeln und Konzentrationen getestet und evaluiert. Darüber hinaus wurde die Fähigkeit der Trennsäule zur Trennung zweier unterschiedlich großer Nanopartikelsupensionen demonstriert. Eine vollsätndige Trennung konnte mit dem Einsäulensystem jedoch nicht erreicht werden. Daher stellt ein Mehrsäulensystem mit den untersuchten Trennsäulen einen vielversprechenden Ansatz für die selektive Größentrennung magnetischer Nanopartikel dar.  
