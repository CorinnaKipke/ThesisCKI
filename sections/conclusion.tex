%% LaTeX2e class for student theses
%% sections/conclusion.tex
%% 
%% Karlsruhe Institute of Technology
%% Institute for Program Structures and Data Organization
%% Chair for Software Design and Quality (SDQ)
%%
%% Dr.-Ing. Erik Burger
%% burger@kit.edu
%%
%% Version 1.1, 2014-11-21

\chapter{Conclusion and Outlook}
\label{ch:Conclusion}
\newacronym{smb}{SMB}{simulated moving bed chromatography}


In the course of this thesis a two-dimensional numerical model of the retention of nanoparticles within magnetized packed beds was developed using the commercially available COMSOL multiphysics software. The main goal of the simulation was to investigate the influences of different parameters like the particle size, flow velocity or magnetic field strength on the retention behaviour of the magnetic nanoparticles. 
A single wire model for HGMS provided by A. Ebner was adapted to the experimental setup used in this thesis \cite{choomphon2017simulation}. This numerical model was validated for the magnetic field and the fluid flow characteristics using analytical and experimental data as well as results from other simulations (see Section \ref{subsec:mod_val}). In addition, the computation time and accuracy of the finite element based model were optimized by a mesh refinement study. For the single wire model constant velocity and periodic boundary conditions were tested on the channel walls. The periodic boundary conditions resulted in more realistic fluid flow conditions found in a packed bed as compared to the constant velocity boundary. Parameter studies, investigating the influence of the magnetic flux density, wire radius, inlet flow velocity and particle size on the residence time of magnetic nanoparticles, were conducted (see Section\,\ref{subsec:Param_res}). It could be shown that an increase in the magnetic flux density leads to an increased retention of the nanoparticles in the vicinity of the ferromagnetic wire. An increase in the wire radius results in a decrease of the magnetization of the wire at the same flux density and therefore lower residence times. Consequently, matrix material with a radius of 30\,nm or smaller should be applied for practical use. Higher flow velocities lead to a higher retention as the particle streamlines are more turbulent and the particles are brought closer to the wire as compared to slower flow velocities. The magnetic force affecting the nanoparticles increases with increasing particle size, which also leads to a stronger retention of the nanoparticles. This indicates a size separation effect for differently sized nanoparticles by their different retention time in magnetized packed beds. The size separation effect is mainly caused by the dependency of the magnetic force on the cube of the particle radius, which leads to larger particles being more affected by the magentic field gradient than smaller particles.

As a column matrix consists of a large number of elements, two and four wire models were developed (see Section \ref{subsubsec:Two_wires}). Again parameter studies were run to investigate the influence of the magnetic flux density, particle size and inlet flow velocity. It could be shown that the inlet flow velocity did not have a significant effect on the particle residence time, in contrast to the single-wire model, as the fluid velocity was highly increased in the cross sections between the wires independently from the inlet velocity. An increasing size separation effect could be demonstrated for increasing magnetic flux densities. However, the model proved to be unstable and did not converge for magnetic flux densities above 35\,mT, where too many particles were retained. The instabilities occurred once the magnetic force affecting the particles became too strong, e.g. for larger nanoparticles over 50\,nm at magnetic flux densities higher than 15\,mT. Therefore, a more stable model design would be desirable, in order to further investigate the influencing parameters. Eventually the whole packed bed should be simulated in all three dimensions to reflect the real situation within a separation column and to reliably predict the retention behaviour of the nanoparticles. This could be achieved by implementing the model in the open-source program OpenLB based on the Lattice Boltzmann method. OpenLB allows a high degree of parallelization, which is necessary for the computation of such a complex model.     

In the second part of this study retention experiments with a separation column were conducted. The objective was to investigate the influence of the magnetic field strength and particle size on the retention behaviour of magnetic nanoparticles in a real packed bed and to confirm the findings from the parameter studies conducted in the first part of this thesis. Furthermore, the size separation effect within the column observed by \cite{AndreMaster} should be reproduced with a different matrix material. 
In a first step, several matrix materials and nanoparticle suspensions were characterized. The matrix materials PMMA and TruForm\textsuperscript{TM} 316-3  were chosen due to their diamagnetic and ferromagnetic properties, respectively, as well as for their size and shape. Almost spherical matrix particles with an average size of 30\,nm or under were desired. The size distribution and the superparamagnetic behaviour of the nanoparticle suspensions were examined. Nanoparticles from Chemagen consisting of a core of magnetite coated with polyvinyl alcohol and plain nanomag\textsuperscript{\textregistered}-D-spio particles from Micromod were selected for the experiments. The Chemagen particles exhibited a much wider particle size distribution than the Micromod particles. In order to optimize the peak shape, a flow rate of 0.5\,ml/min was selected. The external magnetic field was generated by a Helmholtz coil through a square wave signal. This allowed the investigation of the influence of different magnetic field strengths, settings and frequencies. The PMMA matrix material was selected due to its non-magnetic behaviour as a negative control. It could be shown that even for magnetic flux densities as high as 13.4\,mT no significant retention or peak broadening effect occurs. This demonstrates that the retention observed with the ferromagnetic matrix material is solely due to the magnetized packed bed and not produced by the column packing or setup. Furthermore, no size exclusion effect on the nanoparticle suspension through the packed bed could be observed. In contrast, increasing retention and peak broadening could be observed with the ferromagnetic TruForm matrix material for increasing magnetic flux densities. For magnetic field densities larger than 10\,mT a decrease of the peak area was apparent in addition to the retention. For high magnetic field strengths, the larger particles are collected by the matrix and do not elute. Due to the remanence of the matrix material, the particles could only be retrieved after the demagnetization of the packed bed. A maximum retention of 24\,\% was reached with a constant magnetic field and a magnetic flux density of 17\,mT for a 1:150 dilution of the Chemagen nanoparticle suspension.

The size separation effect observed by \cite{AndreMaster} could also be reproduced for the TruForm matrix material with the Chemagen nanoparticles. To evaluate this effect, the resulting peaks of the column experiments were fractionated. The fractions were then analyzed for their particle size distributions in a Zetasizer. The resulting distributions of the peak fractions from the runs with an externally applied magnetic field were compared to the size distribution at 0\,mT and the original feed suspension. In order to quantify the size separation effect more closely, Micromod nanoparticles with an average size of 100 and 20\,nm were used. For the 100\,nm particles, an increase in the D90 value of 34\,\% from the first to the third fraction could be observed. For the 20\,nm however the trend was less pronounced. This could also be due to the particles being close to the detection limit of the measurement system and the large deviations of up to 100\,nm in average particle size specified by the manufacturer. As a last step, the size separation of a 1:1 mixture of the two Micromod nanoparticle suspensions was attempted. Due to the wide particle size distribution of the 20\,nm particles, no clear separation could be measured, even though the usual size separation effect was observed. The resulting chromatogram, however, showed two peaks which became more distinct with an increasing magnetic flux density. This suggests that a complete separation of the components should be possible with either higher magnetic field strengths or a longer separation column. Instead of a longer column also a \gls{smb} could be applied to separate the two peaks. \Gls{smb} is a continuous countercurrent chromatography method, which was developed in the late 1950s by engineers from Universal Oil Products\cite{broughton1961continuous,carson1962rotary}. With this method a continuous binary separation of a mixture can be achieved with high purity and yield. For the \gls{smb} process at least four columns are necessary. This would also allow the possibilities of applying different magnetic field strengths on the different columns to refine the separation process and can lead eventually to a continuous size fractionation process for magnetic nanoparticles at an industrial scale.\\

Overall a retention and size separation effect due to the magnetic field could be demonstrated in both the simulation runs and the laboratory experiments. As predicted by the model an increase in the particle size lead to an increased retention for higher magnetic field strengths in the experimental setup. It was shown in the simulation results that smaller matrix elements produced a higher magnetic field gradient, which in turn resulted in longer residence times. Comparing the results from \cite{AndreMaster} and this work this influence of the matrix particle size could also be observed. Furthermore a proof of principle by a buffer exchange and a negative control with non-magnetic matrix material were successfully performed with the separation colunm. The complete separation of two different sized nanoparticle suspensions, however, was no possible with the one column setup. This could be achieved by extending the experiments to the multicolumn system.     